
Use of the Structural Equations Model in Assessing the Reliability of a New Measurement Technique

Gabrielle E. Kelly

Applied Statistics (1985) vol. 34 no. 3 pp 258- 263

Introduction
The Structural equations model is used to estimate the linear relationship between new and standards method.

The Delta method is used to find the variance of the estimated parameters.

Reply
Bland Altman 

%------------------------------------------------------------------------------------%

\section{Structural Equation modelling}

This is a statistical technique used for testing and estimating causal relationships using a combination of statistical data and qualitative causal assumptions

This technique was proposed by Gabrielle Kelly as a method of assessing the reliability of a new measurement technique.

In this paper, the SEM method is used to assess the linear relationship between the new method and the standard method.

Structural analysis is a generalization of regression analysis.

In Hopkins papers, a critique of the bland Altman plot he makes the following remark

What's needed for a comparison of two or more measures is a generic approach more powerful even than regression to model the relationship and error structure of each measure with a latent variable representing the true value.  

( He also adds that he himself is collaborating in research utilising SEM and Mixed Effects modelling)
\section{Structural Equations}

Kelly proposed the ‘Structural Equation method’ [Kelly 1985]

Altman and Bland [1987]criticize it for a reason that should come as no surprise: 
Knowing the data are consistent with a structural equation with a slope of 1 says something 
about the absence of bias but nothing about the variability about Y = X (the difference between the measurements), which, as has already been stated, is all that really matters.

%-------------------------------------------------------------%
\subsection{Steps of Structural Equation modelling}

1. Model Specification
We must state the theoretical model either as a set of equations.

2.Identification 
This step involves checking that the model can be estimated with observable data, both in theory and in practice.

3.Estimation
The models parameters are statistically estimated from data. (multiple regression is one such method)

4. Model Fit
The estimated model parameters are used to predict the correlations and covariance between measured variables 
The predicted correlations, or covariance are compared to the observed correlations, or covariance. (Measures of model fit are calculated)

%-------------------------------------------------------------%
\subsection{Bland and Altmans Critique}

Does not assess alternative statistical approaches
It is unnecessary to perform elaborate statistical analysis
There are two aspects of agreement that must be considered
•	Bias – describing the average agreement
•	Individual variability

%-------------------------------------------------------------%

Dr Kelly has considered only the first of these.
It is , however, of no clinical value to know that two methods agree on average if one has no idea of the between subject variability.

The Structural equations approach has merit in that it avoids the biased estimation of the slope and intercept that occurs in ordinary least square regression.

“however, it is quite wrong to argue solely from a lack of bias that two methods can be regarded as comparable”

It is, however, the variability around Y=X that is of major interest



%-------------------------------------------------------------%
\subsection{Lewis’s test} 
While regarding a comparison of two pump meters under operational conditions

‘..It is suspected that the various assumptions made by each method are weak under operational conditions’
Lewis listed several sources of variation that relate to the practical aspects of each measurement method.

‘There is little reasons to believe that the laboratory conditions of the devise provide a suitable basis for the conversion of data gathered under operational conditions.


%-------------------------------------------------------------%
Latent variables are variables that are not measured (i.e. not observed) but whose values is observed from other observed variables. One advantage of using latent variables is that it reduces the dimensionality of data. A large number of observable variables can be aggregated in a model to represent an underlying concept, making it easier for humans to understand the data.	[wikipedia]



