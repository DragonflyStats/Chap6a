\documentclass[12pt, a4paper]{article}
\usepackage{natbib}
\usepackage{vmargin}
\usepackage{graphicx}
\usepackage{epsfig}
\usepackage{subfigure}
%\usepackage{amscd}
\usepackage{amssymb}
\usepackage{amsbsy}
\usepackage{amsthm, amsmath}
%\usepackage[dvips]{graphicx}
\bibliographystyle{chicago}
\renewcommand{\baselinestretch}{1.8}
% left top textwidth textheight headheight % headsep footheight footskip
\setmargins{3.0cm}{2.5cm}{15.5 cm}{23.5cm}{0.5cm}{0cm}{1cm}{1cm}
\pagenumbering{arabic}


\begin{document}
\author{Kevin O'Brien}
\title{General Least Squares}
\date{\today}
\maketitle
\tableofcontents \setcounter{tocdepth}{2}

\newpage
\section{Simplifying GLS (K Hayes)}

\subsection{Introduction}

Hayes and Haslett (1998) present an approach to the problem of \textbf{general least squares} estimation of the general linear model in terms of constrained optimization, which is in turn solved via Lagrange multipliers. The crux of the proposed approach is that one system of equations is sufficiently versatile, and provides for \begin{itemize} \item the estimation of new observations, \item estimation of fixed parameters in regression \item estimation of fixed and random effects in mixed models,\item the diagnostics associated with conditional and marginal residuals \item and of subset deletion. \end{itemize}

\subsection{Overview}
Hayes and Haslett (1998) have demonstrated how the problem of best linear unbiased estimation can be posed in terms of Lagrange multipliers. Both BLUE and BLUP can be treated as distinct estimation problems from the following equation.

\begin{equation}
\left(  \begin{array}{cc} V & X \\    X^t & 0 \\
  \end{array}\right)\left(  \begin{array}{c}    \boldsymbol{\lambda}_{z}\\   \boldsymbol{\gamma}_z \\  \end{array}
\right)=\left(  \begin{array}{c}    \mbox{cov}(Y,Z)\\   A^{t} \\  \end{array}\right)\end{equation}


Hence BLUE and BLUP can be considered as the estimation of two different variables from $Y$. This equation has a natural role in the derivation of \emph{leave- k-out} residuals and diagnostic measures, and replaces the traditional approach of using a variety of clumsy updating formulas. Note that this approach may be used to determine the impact of deletion on any quantity computed from $Y$.

\chapter{General Linear model}
\section{General Linear model} Mixed Effects Models are seen as
especially robust in the analysis of unbalanced data when compared
to similar analyses done under the General Linear Model framework
(Pinheiro and Bates, 2000).

A Mixed Effects Model is an extension of the General Linear Model
that can specify additional random effects terms

\subsection{Equivalence of LME model}
Henderson's mixed model equations are presented on page 147 of
Youngjo et al. Youngjo et al demonstrate that this formulation is
equivalent to an augmented general linear model.

Youngjo et al show that the linear mixed effects model can be
shown to be the augmented classical linear model involving fixed
effects parameters only.

\section{Augmented GLMs}
%Augmented General linear models.
% Youngjo et al page 154
\subsection{Augmented linear model}
The subscript $M$ is a label referring to the mean model.
\begin{equation}
\left(%
\begin{array}{c}
  Y \\
  \psi_{M} \\
\end{array}%
\right) = \left(
\begin{array}{cc}
  % after \\: \hline or \cline{col1-col2} \cline{col3-col4} ...
  X & Z \\
  0 & I \\
\end{array}\right) \left(%
\begin{array}{c}
  \beta \\
  \nu \\
\end{array}%
\right)+ e^{*}
\end{equation}




The error term $e^{*}$ is normal with mean zero. The variance
matrix of the error term is given by
\begin{equation}
\Sigma_{a} = \left(%
\begin{array}{cc}
  \Sigma & 0 \\
  0 & D \\
\end{array}%
\right).
\end{equation}

\begin{equation}
X = \left(%
\begin{array}{cc}
  T & Z \\
  0 & I \\
\end{array}%
\right)
\delta = \left(%
\begin{array}{c}
  \beta  \\
  \nu  \\
\end{array}%
\right)
\end{equation}



\begin{equation}
y_{a} = T \delta + e^{*}
\end{equation}

Weighted least squares equation


% Youngjo et al page 154
\newpage





\addcontentsline{toc}{section}{Bibliography}

\bibliography{2012bib}

\end{document} 