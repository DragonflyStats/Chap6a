\documentclass[Chap3bmain.tex]{subfiles}
\begin{document}
	
	\subsection*{Coverage probability (CP)}
	Another user friendly measure of agreement which is related to the computation of the TDI is the so called coverage probability (CP) [11,12]. 
	The CP describes the proportion captured within a pre-specified boundary of the absolute paired-measurement differences from two devices, i.e., the value of p$\kappa$ such that P(|D| < $\kappa$) = p$\kappa$. Therefore one can find p$\kappa$ for a specified boundary $\kappa$ using standard methods for computing probability quantities under normal assumptions [11]:
	
	(13)
	and to obtain a CP estimate, p$\kappa$ can be computed by replacing $\mu_D$ and $\sigma_D$ by their REML estimate counterparts derived from model (1).
	
	As with the TDI, the CP criterion can also be translated into a hypothesis test specification. 
	In this case the interest is to ensure that a specified boundary of the absolute paired-measurement differences captures at least a predetermined proportion, p0:
	
	
	The proposed TI method for inference about the TDI can be utilized to perform inferences about the CP estimates. From the TI in (10) it follows that
	
	(14)
	Now $\kappa$ is a fixed known boundary, and our interest lies in finding a lower confidence bound for the CP estimate. 
	Thus, one can find a lower confidence bound for a non-central Student-t proportion with confidence level 1 - $\alpha$ by searching the non-centrality parameter, 
	that depends on  and hence on p$\kappa$, that satisfies
	
	(15)
	and once the non-centrality parameter  is achieved, a lower bound about the proportion p$\kappa$ is found using equation (5), 
	
	% p$\kappa$ = Φ() - Φ(-2μD/σD - ).
	
	However, the non-centrality parameter cannot be found in a closed form, so one may use again a modified version of the binary search algorithm as follows:
	
	\begin{enumerate}
		\item begin with the interval [low = 0; high = 1], as p$\kappa$ is bounded by the interval (0,1);
		
		\item calculate the midpoint of the interval \textit{mid = (low + high)/2} and compute the difference ;
		
		\item if d is greater than 0 up to a tolerance bound $\delta$ (i.e., ), then recalculate the interval [low = mid + $\delta$; high = 1]; if it is 
		lower than 0 up to a tolerance bound $\delta$ (i.e. ), then recalculate the interval [low = 0; high = mid - $\delta$];
		
		\item repeat steps 2-3 until convergence, i.e. until d satisfies .
	\end{enumerate}