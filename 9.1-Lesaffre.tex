\documentclass[Main.tex]{subfiles}
\begin{document}
	\newpage
%---------------------------------------------------------%
	\section{Lesaffre's paper.}
	Lesaffre considers the case-weight perturbation approach.
	
	
	\citep{cook86}
	Cook's 86 describes a local approach wherein each case is given a
	weight $w_{i}$ and the effect on the parameter estimation is
	measured by perturbing these weights. Choosing weights close to
	zero or one corresponds to the global case-deletion approach.
	
	Lesaffre  describes the displacement in log-likelihood as a useful
	metric to evaluate local influence %\citep{cook86}.
	
	%\citet{lesaffre}
	Lesaffre describes a framework to detect outlying observations
	that matter in an LME model. Detection should be carried out by
	evaluating diagnostics $C_{i}$ , $C_{i}(\alpha)$ and $C_{i}(D,
	\sigma^2)$.
	
	Lesaffre defines the total local influence of individual $i$ as
	\begin{equation}
	C_{i} = 2 | \triangle \prime _{i} L^{-1} \triangle_{i}|.
	\end{equation}
	
	The influence function of the MLEs evaluated at the $i$th point
	$IF_{i}$, given by
	\begin{equation}
	IF_{i} = -L^{-1}\triangle _{i}
	\end{equation}
	can indicate how $\hat{theta}$ changes as the weight of the $i$th
	subject changes.
	
	The manner by which influential observations
	distort the estimation process can be determined by inspecting the
	interpretable components in the decomposition of the above
	measures of local influence.
	
	
	Lesaffre comments that there is no clear way of interpreting the
	information contained in the angles, but that this doesn't mean
	the information should be ignored.
	%-----------------------------------------------------------------------------------------%
	\newpage
\section{Lesaffre's paper.} %5.6

\begin{itemize}
\item Lesaffre considers the case-weight perturbation approach.




\item \citep{cook86} describes a local approach wherein each case is given a weight $w_{i}$ and the effect on the parameter estimation is measured by perturbing these weights. Choosing weights close to zero or one corresponds to the global case-deletion approach.


\item Lesaffre  describes the displacement in log-likelihood as a useful metric to evaluate local influence \citep{cook86}.




\item \citet{lesaffre} describes a framework to detect outlying observations that matter in an LME model. Detection should be carried out by evaluating diagnostics $C_{i}$ , $C_{i}(\alpha)$ and $C_{i}(D,\sigma^2)$.
\end{itemize}






Lesaffre defines the total local influence of individual $i$ as
\begin{equation}
C_{i} = 2 | \triangle \prime _{i} L^{-1} \triangle_{i}|.
\end{equation}




The influence function of the MLEs evaluated at the $i$th point $IF_{i}$, given by
\begin{equation}
IF_{i} = -L^{-1}\triangle _{i}
\end{equation}
can indicate how $\hat{theta}$ changes as the weight of the $i$th
subject changes.


The manner by which influential observations distort the estimation process can be determined by inspecting the
interpretable components in the decomposition of the above measures of local influence.




Lesaffre comments that there is no clear way of interpreting the information contained in the angles, but that this doesn't mean the information should be ignored.




\bibliography{DB-txfrbib}
\end{document}