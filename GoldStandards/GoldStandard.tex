\documentclass[Chap1bmain.tex]{subfiles}
\begin{document}
\citet[p.47]{DunnSEME} cautions that`gold standards' should not be
assumed to be error free. `It is of necessity a subjective
decision when we come to decide that a particular method or
instrument can be treated as if it was a gold standard'. The
clinician gold standard , the sphygmomanometer, is used as an
example thereof.  The sphygmomanometer `leaves considerable room
for improvement' \citep{DunnSEME}. \citet{pizzi} similarly
addresses the issue of glod standards, `well-established gold
standard may itself be imprecise or even unreliable'.


The NIST F1 Caesium fountain atomic clock is considered to be the
gold standard when measuring time, and is the primary time and
frequency standard for the United States. The NIST F1 is accurate
to within one second per 60 million years \citep{NIST}.

Measurements of the interior of the human body are, by definition,
invasive medical procedures. The design of method must balance the
need for accuracy of measurement with the well-being of the
patient. This will inevitably lead to the measurement error as
described by \citet{DunnSEME}. The magnetic resonance angiogram,
used to measure internal anatomy,  is considered to the gold
standard for measuring aortic dissection. Medical test based upon
the angiogram is reported to have a false positive reporting rate
of 5\% and a false negative reporting rate of 8\%. This is
reported as sensitivity of 95\% and a specificity of 92\%
\citep{ACR}.

In literature they are, perhaps more accurately, referred to as
`fuzzy gold standards' \citep{phelps}. Consequently when one of the methods is
essentially a fuzzy gold standard, as opposed to a `true' gold
standard, the comparison of the criterion and test methods should
be consider in the context of a comparison study, as well as of a
calibration study.

\section{Gold and Bronze Standards}

\noindent \citet[p.47]{DunnSEME} cautions that 'gold standards' should not be assumed to be error free. \emph{It is of necessity a subjective decision when we come to decide that a particular method or instrument can be treated as if it was a gold standard}.
The clinician gold standard , the sphygmomanometer, is used as an example thereof.  The sphygmomanometer\emph{leaves considerable room for improvement} \citep{DunnSEME}. \citet{pizzi} similarly addresses the issue of glod standards: \emph{well-established gold standard may itself be imprecise or even unreliable}.


The NIST F1 Caesium fountain atomic clock is considered to be the gold standard when measuring time, and is the primary time and frequency standard for the United States. The NIST F1 is accurate
to within one second per 60 million years. \citep{NIST}.

Measurements of the interior of the human body are, by definition, invasive medical procedures. The design of method must balance the need for accuracy of measurement with the well being of the patient. This will inevitably lead to the measurement error as described by \citet{DunnSEME}. The magnetic resonance angiogram ,used to measure internal anatomy,  is considered to the gold standard for measuring aortic dissection. Medical test based upon the Angiogram is reported to have a false positive reporting rate of 5\% and a false negative reporting rate of 8\%. ( This is
reported as sensitivity of 95\% and a specificity of 92\%)
\citep{ACR}

In literature they are, perhaps more accurately, referred to as
'bronze standards'. Consequently when one of the methods is
essentially a bronze standard, as opposed to a `true' gold
standard, the comparison of the criterion and test methods should
be consider in the context of a comparison study, as well as of a
calibration study.



\subsection{Fuzzy Gold Standards} The Gold Standard is considered to be the most
accurate measurement of a particular parameter. But even gold
standard raters must be assumed to have some level of measurement
error. Fuzzy gold standard are considered by Phelps and Hutson (
1994)

%%%%%%%%%%%%%%%%%%%%%%%%%%%%%%%%%%%%%%%%%%%%%%%%%%%%%%%%%%%%%%%%%%%%%%%%%%%%%%%%%%%%%%%%%%%%%%%%%%%%%%%%
\section{Fuzzball Agreement}
Fuzzball agreement is a case where the correlation coefficient is close to zero. The sample values is restricted to a narrow range. but an examination of a relevant scatter-plot would indicate that
there is agreement between the two methods.
\\
Agreement - a numerical measure Hutson et al define a numerical measure for agreement.
\\
For example, suppose the pairs of rater measurements are (1, 1), (1.1, 1), (1, 1.1), and (1.1, 1.1) then the sample Pearson correlation r = .0, yet the two raters or devices are considered to be in good agreement. We will refer to the instance where r is close to 0, yet there may be good agreement as "fuzzball agreement." \\Fuzzball agreement occurs quite often in practice when the sample values have very narrow or restricted ranges. Fuzzball agreement is just one instance where the correlation coefficient is a poor measure of agreement. \\Furthermore, note that the ICC is also a poor measure of agreement when there is fuzzball agreement. At the other extreme suppose the same raters given in the previous example had pairs of measurements (1, 101), (2, 102), (3, 103), and (4, 104) on the same relative scale as before. In this instance, r = 1.0, yet there is large disagreement between rater.

\section{Types of Method Comparisons} \citet{Lewis} categorize
method comparison studies into three different types, with the
first two being of immediate concern. A method that is not considered to be a gold standard is referred
to as an 'approximate method'.

\textbf{1. Calibration problems}. The purpose is to establish a
relationship between methods, one of which is an approximate
method, the other a gold standard. The results of the approximate
method can be mapped to a known probability distribution of the
results of the gold standard.

\smallskip
\textbf{2. Comparison problems} - When two approximate methods,
that use the same units of measurement, are to be compared.

\smallskip
\textbf{3. Conversion problems} -  When two approximate methods,
that use different units of measurement, are to be compared. This
situation would arise when the measurement methods use 'different
proxies', i.e different mechanisms of measurement.

\bigskip
\citet{DunnSEME} makes two important points in relation to these
categories. Firstly he remarks that there isn't clear cut
differences between each category.

Secondly he comments on the clinician gold standard, the
sphygmomanometer, \emph{leaves considerable room for improvement}.
\citet{pizzi} also attends to this issue: \emph{well-established
	gold standard may itself be imprecise or even unreliable}.
\bigskip
The Magnetic resonance angiogram is considered to the gold
standard for measuring aortic dissection, with a sensitivity of
95\% and a specificity of 92\% . \citep{ACR}
\bigskip
In literature they are, perhaps more accurately, referred to as
'bronze standards'.
\bigskip

Consequently when one of the methods is essentially a bronze
standard, as opposed to a true gold standard, the comparison
procedure should be considered as being of the second category.

\end{document}
